Se ha hecho notar en anteriores capítulos que a pesar de no realizarse un
plan de pruebas exhaustivo y perfectamente documentado para cada parte del
desarrollo, la estrecha colaboración con el cliente ha hecho posible probar cada
evolución del código en un entorno prácticamente real.

Por supuesto, antes de informar al cliente de cualquier progreso en el
desarrollo, el proyectante realizaba pruebas básicas: desde que la interfaz
gráfica era perfectamente visible en los principales navegadores (Internet
Explorer 7+, Mozilla Firefox y Google Chrome, el empleado de forma preferida en
la empresa), hasta que los datos se guardaban y recuperaban correctamente de la
base de datos, pasando por la correcta solicitud de datos obligatorios y buen
comportamiento con datos de entrada inesperados.

De este manera, el módulo de personal fue probado por los empleados de
Ingeniería e Innovación a modo de registro de empleados --todavía no podían
asignarse horas ni guardar datos sobre actividades--, desde mediados de
agosto hasta diciembre del 2010. Aunque la funcionalidad era todavía muy básica,
el registro de los datos anuales de los empleados resultó útil por la
centralización de este tipo de información.

Similarmente, antes de que fuera posible conectar las actividades con los
empleados, se introdujeron datos de las actividades (planes de trabajo) de
proyectos concluidos a modo de testeo general, hecho del que se encargó
principalmente el proyectante durante su periodo de prácticas en la empresa.

Finalmente, una vez integrado todo el sistema, este empezó a utilizarse de
modo casi inmediato, ya que hubo que comenzar a justificar los primeros
proyectos con la nueva metodología requerida por la ADER, que se
describió con cierto detalle en la sección  \ref{sec:necesidad} al discutir la
necesidad del proyecto.

Apenas un mes después de que la herramienta comenzase a utilizarse en casos
reales y actuales y se hubieran reunido algunas sugerencias de modificación,
el proyectante sufrió una indisponibilidad familiar entre enero y febrero de
2011, periodo durante el cual cesó cualquier actividad del proyectante en
relación con la herramienta. Sin embargo, Ingeniería e Innovación siguió
utilizando el sistema sin mayores problemas.

\textit{Sin mayores problemas} no quiere decir \textit{sin problemas}, y hubo
uno en particular que, sin ser especialmente molesto, resultó de gran interés
para el proyectante hasta el punto de realizar la prueba que se detalla en la
siguiente sección.

\section{Prueba de distribución de horas y \textit{bug} horario}

El problema observado es que al pedirle a la herramienta que distribuyera un
número de horas en un espacio de tiempo, había ocasiones ($\sim 5\%$, como se
comprobará) en que no se distribuían todas las horas, tal y como uno esperaría.
Había una diferencia entre lo que se pedía distribuir y el resultado de la
distribución.

Se ha comentado anteriormente que la aplicación corrige los excesos a nivel del
proyecto: si se intenta asignar 300 horas a un empleado en un mismo proyecto en
un mes, esa cifra se va a reducir al número de días laborables del mes
multiplicadas por ocho. No es esa la fuente del error, pues es trivial, sino
que este se observó incluso cuando, en teoría, debería haber sido posible una
asignación total.

Dado que el algoritmo de distribución de horas (véase la sección
\ref{sec:algoritmo_distribucion} para recuperar los detalles) redondea al alza a
múltiplos de 5 cuando es posible, tiene en cuenta el número de días laborables
del mes..., resultaba difícil encontrar un error evidente en el código y
también datos de entrada que dieran error, por
lo que se escribió el \textit{script} del Apéndice
\ref{apx:script_distribucion}.

Los datos para el test son generados automáticamente, y se han
considerado suficientemente aleatorios dada la repetición periódica de los
calendarios (hay 14 distintos: el año puede comenzar en cualquiera de los 7 días
de la semana, y puede ser bisiesto o no):

\begin{lstlisting}
$mes_inicio = rand(1,12);
$anio_inicio = rand(2010,2025);
$dia_inicio = rand(1,date('t',mktime(0,0,0,$mes_inicio,1,$anio_inicio)));
$fecha_inicio = mktime(0,0,0,$mes_inicio,$dia_inicio,$anio_inicio);
$fecha_fin = mktime(0,0,0,$mes_inicio,$dia_inicio + rand(0,800),$anio_inicio);
$duracion_en_dias = laborables($fecha_inicio,$fecha_fin);
$duracion_en_horas = $duracion_en_dias * rand(1,900) / 100;
\end{lstlisting}

En el cuadro \ref{cua:resultados_script} se
muestran los resultados tras ejecutar el script original 10 veces.

\begin{table}
\centering
\footnotesize
\begin{tabular}{|c|c|c|c|}\hline
\textbf{Prueba} $\bf{n}$ & \textbf{Iteraciones} & \textbf{Errores} &
\textbf{\% Errores} \\\hline\hline
1 &  50 &   0 & 0\% \\\hline
2 &  50 &   4 & 8\% \\\hline
3 &  50 &   1 & 2\% \\\hline
4 &  50 &   3 & 6\% \\\hline
5 &  50 &   1 & 2\% \\\hline
6 &  50 &   4 & 8\% \\\hline
7 &  50 &   1 & 2\% \\\hline
8 &  50 &   5 & 10\% \\\hline
9 &  50 &   3 & 6\% \\\hline
10 &  50 &   3 & 6\% \\\hline
11 &  50 &   5 & 10\% \\\hline
12 &  50 &   2 & 4\% \\\hline
13 &  50 &   0 & 0\% \\\hline
14 &  50 &   4 & 8\% \\\hline
15 &  50 &   2 & 4\% \\\hline
16 &  50 &   2 & 4\% \\\hline
17 &  50 &   4 & 8\% \\\hline
18 &  50 &   1 & 2\% \\\hline
19 &  50 &   0 & 0\% \\\hline
20 &  50 &   3 & 6\% \\\hline
21 &  50 &   2 & 4\% \\\hline
22 &  50 &   1 & 2\% \\\hline
23 &  50 &   0 & 0\% \\\hline
24 &  50 &   2 & 4\% \\\hline
25 &  50 &   3 & 6\% \\\hline
26 &  50 &   2 & 4\% \\\hline
27 &  50 &   1 & 2\% \\\hline
28 &  50 &   2 & 4\% \\\hline
29 &  50 &   2 & 4\% \\\hline
30 &  50 &   4 & 8\% \\\hline\hline
\textbf{Total} &  \textbf{1500} &  \textbf{67} & \textbf{4.467\%} \\\hline
\end{tabular}
\caption{Prueba del algoritmo original.}
\label{cua:resultados_script}
\end{table}

A continuación, un resultado generado aleatoriamente que daba error:

\begin{itemize}
\item Fecha de inicio: 15/02/2011
\item Fecha de fin: 11/04/2012
\item Horas límite: 2368
\item Duración en horas: 2368
\item Horas por día: 8
\item Horas asignadas: 2360 
\item ¿Éxito?: No!
\item Diferencia: -8
\end{itemize}

Tras probar estos valores con una asignación en la herramienta, se vio que los
problemas venían siempre en los meses de marzo y octubre. «¡Eureka! El problema
es el cambio de hora». Al comprobar día a día si era laborable o no, la función
que se describió en la sección \ref{sec:algoritmo_distribucion} pasaba al día
siguiente de la siguiente forma:

\begin{lstlisting}
$current_date += 86400; //24*60*60=86400
\end{lstlisting}

Sin embargo, los días del cambio de hora NO tienen 24 horas, y PHP lo sabe, de
manera que se creaban errores cuando la exigencia de horas era máxima
(dedicación del empleado a jornada completa). La solución, pasar al día
siguiente de la siguiente manera:

\begin{lstlisting}
$current_date = 
	mktime(0,0,0,
		date('m',$current_date),
		date('d',$current_date)+1,
		date('Y',$current_date)
	);
\end{lstlisting}

Contribuía de manera vital al error que tanto las fechas de inicio como de fin
se pasaban en formato dd/mm/aaaa 00:00. Si se hubiera usado la notación
dd/mm/aaaa 23:59 para la fecha de fin, tampoco hubiese surgido el problema.

El proyectante ha ejecutado el \textit{script} modificado y una vez superadas
las 2000 pruebas no se había encontrado ningún error, es decir, siempre se han
guardado tantas horas como se había requerido, después de corregir a los
límites máximos mensuales.



