\section*{MySQL-Front}

Predecesor de HeidiSQL, MySQL-Front es un cliente ligero de \textit{front-end}
de MySQL desarrollado por Ansgar Becker para entornos Windows. Es un cliente del
que se pueden esperar todas las funciones básicas y algunas avanzadas en
relación con la gestión de bases de datos, incluida su exportación, pero su
evolución y desarrollo es constante con objeto de implementar la funcionalidad
más avanzada.

La razón por la que se ha empleado este programa es que Ingeniería e Innovación
lo empleaba para la gestión \textit{front-end} del resto de sus bases de datos.

Enlace de interés: \href{http://www.heidisql.com/}{www.heidisql.com}

\section*{Geany}

Geany es uno de tantos editores de texto ligeros con soporte para multitud de
lenguajes, entre los que se encuentran, por supuesto, PHP, HTML, JavaScript y
CSS, los empleados en el desarrollo del presente proyecto. Su principal
desarrollador es Enrico Tröger, y fue lanzado en 2005.

El motivo principal de su elección es su cualidad multiplataforma (el
desarrollo se ejecutó tanto desde entornos Windows como GNU/Linux), aunque
también posee el resto de cualidades que se pueden esperar: autocompletado,
soporte multidocumento, soporte de proyectos, coloreado de sintaxis y emulador
de terminal incrustado.

Enlace de interés: \href{http://www.geany.org/}{www.geany.org}

\section*{Kile + TeX Live}

TeX Live es una distribución TeX que ha reemplazado a su predecesora teTeX, y
se encuentra presente por defecto en numerosas distribuciones GNU/Linux como
Fedora, Debian o Ubuntu, pero también está disponible para otros sistemas
operativos, incluidos Mac OS X, Solaris y Windows. Su desarrolló fue comenzado
por Sebastian Rahtz en 1996 y en la actualidad es mantenida por Karl Berry.

Kile, por su parte, es un editor de código TeX/LaTeX sobre plataformas tipo
UNIX como Mac OS X y GNU/Linux.

Enlaces de interés: \href{http://kile.sourceforge.net/}{kile.sourceforge.net},
\href{ http://www.tug.org/texlive/}{www.tug.org/texlive}

\section*{Microsoft Visio y Project}

Microsoft Visio es un programa de dibujo vectorial para plataformas Windows.
Permite realizar diagramas de bases de datos, diagramas de flujo de programas,
UML, y más. En esta memoria, se ha empleado para los diagramas de casos de uso,
los diagramas de secuencia y  el diagrama de entidad/relación.

Por su parte, Microsoft Project es un software de administración de proyectos
diseñado, desarrollado y comercializado por Microsoft para asistir a
administradores de proyectos en el desarrollo de planes, asignación de recursos
a tareas, dar seguimiento al progreso, administrar presupuesto y analizar cargas
de trabajo. En esta memoria, sin embargo, simplemente se ha usado para realizar
el diagrama de Gantt.

Enlaces de interés:
\href{http://office.microsoft.com/en-us/visio/}{office.microsoft.com/en-us/visio
}, \newline \href{http://www.microsoft.com/project}{www.microsoft.com/project}

\section*{Gimp}

GIMP (GNU Image Manipulation Program) es un programa multiplataforma de edición
de imágenes digitales en forma de mapa de bits, tanto dibujos como fotografías.
Es un programa libre y gratuito. Forma parte del proyecto GNU y está disponible
bajo la Licencia pública general de GNU. En esa memoria, se ha empleado para la
generación de los tipos de archivo .eps requeridos por LaTeX y para la edición
general de las imágenes.

Enlace de interés: \href{http://www.gimp.org/}{www.gimp.org}
