La presente memoria muestra como se ha desarrollado una herramienta cuya
principal tarea es la de recopilación de datos para su almacenaje y/o su
tratamiento. Ya que la aplicación va a ser usada en España ésta debe de
adaptarse a las leyes españolas. Actualmente existe la Ley Orgánica 15/1999 de
13 de diciembre de Protección de Datos de Carácter Personal (LOPD), que es una
ley que tiene por objeto garantizar y proteger, en lo que concierne al
tratamiento de los datos personales, las libertades públicas y los derechos
fundamentales de las personas físicas, y especialmente de su honor, intimidad y
privacidad personal y familiar. 

Su objetivo principal es regular el tratamiento
de los datos y ficheros, de carácter personal, independientemente del soporte en
el cual sean tratados, los derechos de los ciudadanos sobre ellos y
las obligaciones de aquellos que los crean o tratan. Por todo ello, y dado que
en nuestra aplicación se recaban datos de clientes, proveedores y comisionistas,
esta ley es para nosotros de obligado cumplimiento. 

El órgano de control del
cumplimiento de la normativa de protección de datos dentro del territorio
español, con carácter general es la Agencia Española de Protección de
Datos (AEPD), existiendo otras Agencias de Protección de Datos de
carácter autonómico, en las Comunidades Autónomas de Madrid, Cataluña y en el
País Vasco. 

Los datos personales se clasifican en función de su mayor o menor
grado de sensibilidad, siendo los requisitos legales y de medidas de seguridad
informáticas más estrictos en función de dicho mayor grado de sensibilidad,
siendo obligatorio por otro lado, en todo caso la declaración de los ficheros de
protección de datos a la ``Agencia Española de Protección de Datos''. 

Los interesados a los que se soliciten datos personales deberán ser previamente
informados de modo expreso, preciso e inequívoco:

\begin{itemize}
\item[1)] De la existencia de un fichero o tratamiento de datos de carácter
personal, de la finalidad de la recogida de éstos y de los destinatarios de la
información.

\item[2)] Del carácter obligatorio o facultativo de su respuesta a las preguntas
que les sean planteadas.

\item[3)] De las consecuencias de la obtención de los datos o de la negativa a
suministrarlos.

\item[4)]De la posibilidad de ejercitar los derechos de acceso, rectificación,
cancelación y oposición.

\item[5)] De la identidad y dirección del responsable del tratamiento o, en su
caso, de su representante.
\end{itemize}

Las sanciones por incumplir esta normativa tienen una elevada cuantía, siendo
España el país de la Unión Europea que tiene las sanciones más altas en materia
de protección de datos. Dichas sanciones dependen de la infracción cometida y se
dividen en:

\begin{itemize}
\item Las sanciones leves van desde 601’01 a 60.101’21 
\item Las sanciones graves van desde 60.101’21 a 300.506’05 e
\item Las sanciones muy graves van desde 300.506’05 a 601.012’10 e
\end{itemize}

Se permite sin embargo, el tratamiento de datos de carácter personal sin haber
sido recabados directa mente del afectado o interesado, aunque no se exime de la
obligación de informar de forma expresa, precisa e inequívoca, por parte del
responsable del fichero o su representante, dentro de los tres meses siguientes
al inicio del tratamiento de los datos.

Excepción: No será necesaria la comunicación en tres meses de dicha información
si los datos han sido recogidos de ``fuentes accesibles al público'', y se
destinan a la actividad de publicidad o prospección comercial, en este caso ``en
cada comunicación que se dirija al interesado se le informará del origen de los
datos y de la identidad del responsable del tratamiento así como de los derechos
que le asisten''.

Por todo ello, llevando los requerimientos de dicha ley a nuestro caso, será
necesario tomar las siguientes
medidas:

\begin{itemize}
\item Registro de la base de datos ante la oficina central de la Agencia
Española de Protección de datos.
\item Posibilitar a las personas cuyos datos sean registrados en la aplicación
el ejercitar sus derechos de acceso, rectificación, cancelación y oposición bien
por correo postal, fisicamente en las oficinas o por correo electrónico.
\item Informar a los clientes de la introducción de sus datos en nuestra base de
datos.
\end{itemize}

Esta podría ser una cláusula modelo de información/consentimiento de derechos
amparados por la LOPD a incluir en todos los documentos de la empresa Arnedo \&
Belmonte Ingeniería e Innovación:

\textit{En cumplimiento de la Ley Orgánica 15/1999, de 13 de diciembre de
Protección de Datos de Carácter Personal (LOPD), Arnedo \& Belmonte Ingeniería e
Innovación S.L., como responsable del fichero informa de las siguientes
consideraciones: Los datos de carácter personal que le solicitamos, quedarán
incorporados a un fichero cuya finalidad es mantener un registro de los
clientes de la empresa para su almacenamiento/generación de documentos. Los
campos marcados con asterisco son de cumplimen tación obligatoria, siendo
imposible realizar la finalidad expresada si no aporta esos datos. Queda
igualmente informado de la posibilidad de ejercitar los derechos de acceso,
rectificación, cancelación y oposición, de sus datos personales en la dirección
de correo lopd@ingenieriaeinnovacion.es o bien en la dirección fiscal de la
empresa por correo postal.}
